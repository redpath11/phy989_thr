\documentclass[10pt,showpacs,preprintnumbers,footinbib,amsmath,amssymb,aps,prl,twocolumn,groupedaddress,superscriptaddress,showkeys]{revtex4-1}
\usepackage{graphicx}
\usepackage{dcolumn}
\usepackage{bm}
\usepackage[colorlinks=true,urlcolor=blue,citecolor=blue]{hyperref}
\usepackage{color}
\usepackage{listings}

\newcommand{\deriv}[3][]{% \deriv[<order>]{<func>}{<var>}
	\ensuremath{ \frac{d^{#1} {#2}}{d {#3}^{#1}} } }


\begin{document}
\title{Project 1}
\author{Thomas Redpath and Dayah Chrisman}
\affiliation{Department of Physics, Michigan State University}
\begin{abstract}

We solve the Lippman-Schwinger equation for proton-neutron scattering to obtain the $l=0$
phase shift $\delta _0$ as a function of lab energy. We compare our results to the $\delta _0$
extracted from experiment ...

\end{abstract}
\maketitle

\section{Introduction}




\section{Theory and algorithms}\label{sec:theory}

\subsection{Lippman-Schwinger Equation}

In order to calculate the phase shifts $\delta_l$, we need to solve the Schr{\"o}dinger equation
for the neutron-proton system with $E > 0$. This can be expressed as an integral equation for
the reaction matrix

\begin{equation}
\begin{split}
    R_l(k,k') &= V_l(k,k') + \\
  &\frac{2}{\pi}\hat{P}
                \int_0^{\infty}dqq^2V_l(k,q)\frac{1}{E-q^2/m}R_l(q,k'),
   \label{eq:ls1}
\end{split}
\end{equation}
where $E = k _0 ^2 / m$ is the total kinetic energy of the nucleons in the center-of-mass
system and the momentum-space representation of the potential is used. We solve this
equation numerically on a discretized domain where we've set the mesh points and
weights using Gaussian quadrature. We map the points and weights from the interval of
the Legendre polynomials ($x \in [-1,1]$) to our desired interval $r \in [0,\infty)$ using the
mapping

\begin{align*}
          k_i&=const\times tan\left\{\frac{\pi}{4}(1+x_i)\right\} \\
          \omega_i&= const\frac{\pi}{4}\frac{w_i}{cos^2\left(\frac{\pi}{4}(1+x_i)\right)}.
\end{align*}

We summarize the prescription for solving eq.~\ref{eq:ls1} numerically from
\citet{lectureNotes}. First, we employ Cauchy's principle-value prescription
to re-write eq.~\ref{eq:ls1} in the form

\begin{equation}
\begin{split}
    R(k,k') &= V(k,k') + \\
   & \frac{2}{\pi}
                \int_0^{\infty}dq
                \frac{q^2V(k,q)R(q,k')-k_0^2V(k,k_0)R(k_0,k')  }
                     {(k_0^2-q^2)/m},
   \label{eq:ls2}
\end{split}
\end{equation}
which in the is represented on the discretized mesh as

\begin{equation}
\begin{split}
R(k,k') &= V(k,k') +\frac{2}{\pi}\sum_{j=1}^N\frac{\omega_jk_j^2V(k,k_j)R(k_j,k')}{(k_0^2-k_j^2)/m}- \\
  & \frac{2}{\pi}k_0^2V(k,k_0)R(k_0,k')\sum_{n=1}^N\frac{\omega_n}{(k_0^2-k_n^2)/m},
\end{split}
\label{eq:ls3}
\end{equation}
where there are $N$ mesh points from the Gaussian quadrature mapping
and one point that is the chosen ``observable'' point corresponding to the
relative momentum for which we want to calculate the phase shift. Total
there are $(N+1)$ mesh points.

Next, we organize eq.~\ref{eq:ls3} as a matrix equation by defining a
matrix $A$

\begin{equation*}
	A_{i,j} = \delta _{i,j} - V(k_i,k_j)u_j
\end{equation*}
where

\begin{equation}
     u_j=\frac{2}{\pi}
         \frac{\omega_jk_j^2}{(k_0^2-k_j^2)/m}\hspace{1cm}
         j=1,N
\label{eq:uj}
\end{equation}
and

\begin{equation}
     u_{N+1}=-\frac{2}{\pi}
          \sum_{j=1}^N\frac{k_0^2\omega_j}{(k_0^2-k_j^2)/m}.
\label{eq:u0}
\end{equation}

We can now solve for the $R$ matrix by inverting $A$ and multiplying $V$
by $A^{-1}$. Finally, the $R(N+1,N+1)$ element is related to the phase
shift by

\begin{equation}
	R(k_0,k_0)=-\frac{tan\delta}{mk_0}
	\label{eq:PhaseShift}
\end{equation}
Using this prescription and some model for the interaction (see next
subsection), we can calculate $\delta$ for scattering at some incident energy.
We describe the \texttt{C++} code used to make these calculations in
the Methods section.


\subsection{Potential Model}

We currently have coded two potentials that can be used in the phase shift
calculation. The first is a finite square well (eq.~\ref{eq:FSpot}) used to
benchmark the program against the analytical result. The second is a
parameterized neutron-proton interaction (eq.~\ref{eq:NPpot}).

\begin{equation}
	V = -V_0 \Theta(a - r)
	\label{eq:FSpot}
\end{equation}

\begin{equation}
	V(r)=V_a \frac{e^{-ax}}{x}+V_b \frac{e^{-bx}}{x}+V_c \frac{e^{-cx}}{x},
	\label{eq:NPpot}
\end{equation}
with $x=\mu r$, $\mu=0.7$ fm$^{-1}$ (the inverse of the pion mass),
$V_a=-10.463$ MeV and $a=1$, $V_b=-1650.6$ MeV and $b=4$ and
$V_c=6484.3$ MeV and $c=7$.

For both potentials, we find the momentum-space representations using
the Fourier-Bessel transform, we specialize to the case $l=0$ since we're
only aiming to calculate the $\delta_0$. For the finite well potential we
have

\begin{align*}
	V_l(k,k')&= \int j_l(kr)V(r)j_l(k'r)r^2dr\\
	&= V_0 \int_0^a \frac{\sin(kr) \sin(k'r) r^2}{k k' r^2} dr\\
	&= \frac{V_0}{k k'} \int_0^a \sin(kr) \sin(k'r) dr\\
	&= \frac{V_0}{k k'} \frac{k' \sin(ka) \cos(k'a) - k \cos(ka) \sin(k'a)}{k^2 - k'^2}
\end{align*}
This results in 

\begin{equation}
	V(k,k') = \frac{k_j \sin()\cos() + \cos()\sin()}{}
\end{equation}


\subsection{Reaction Matrix}


%\subsection*{The General Algorithm}

%\[
%\begin{bmatrix}
%	d_1 & a_1 & 0 & \dots & \dots & 0 \\
%	c_1 & d_2 & a_2 & 0 & \dots & 0 \\
%	0 & c_2 & d_3 & a_3 & 0 & \dots \\
%	\vdots & & \ddots & \ddots & \ddots & \vdots \\
%	\vdots & & & c_{n-2} & \ddots & a_{n-1}\\
%	0 & & & & c_{n-1} & d_n \\
%\end{bmatrix}
%~
%\begin{bmatrix}
%	u_1\\
%	u_2\\
%	u_3\\
%	u_4\\
%	\vdots\\
%	u_n\\
%\end{bmatrix}
%~
%=
%~
%\begin{bmatrix}
%	f_1\\
%	f_2\\
%	f_3\\
%	f_4\\
%	\vdots\\
%	f_n\\
%\end{bmatrix}
%\]

\section{Methods}

We implemented our numerical solution to the Lippman-Schwinger equation as a \texttt{C++} class.
This class consists of three two-dimensional data structures to hold the $V$, $A$ and $R$ matrices,
two one-dimensional data structures to hold the weights and momentum mesh points that define the
integration domain and several ancillary variables that specify the nucleon mass and other book-keeping
parameters. Application of the algorithms sketched in the previous section is carried out in a series of
member functions that (a) set up the mesh points and weights (b) set up the potential matrix (c) set
up the $A$ matrix (d) invert $A$ to get $R$ then extract $\delta _0$. In principle,
different functions may be used to set up different potentials; we coded one for the model NN interaction
given above and one for a finite spherical well to check that we recover the analytic result.
This class is defined in the source
file \texttt{NucleonScattering.cpp} and \texttt{NucleonScattering.hh}. An instance of the class is
created and the correct sequence of methods is called in the \texttt{main} function defined in
\texttt{main.cpp}.


%% -------------------------- CODE LISTING -------------------------- %%

%% 
 \definecolor{mygreen}{rgb}{0,0.6,0}
 \definecolor{mygray}{rgb}{0.5,0.5,0.5}
 \definecolor{mymauve}{rgb}{0.58,0,0.82}

 \lstset{ %
   backgroundcolor=\color{white},   % choose the background color; you must add \usepackage{color} or \usepackage{xcolor}
   basicstyle=\footnotesize,        % the size of the fonts that are used for the code
   breakatwhitespace=false,         % sets if automatic breaks should only happen at whitespace
   breaklines=true,                 % sets automatic line breaking
   captionpos=b,                    % sets the caption-position to bottom
   commentstyle=\color{mygreen},    % comment style
   deletekeywords={...},            % if you want to delete keywords from the given language
   escapeinside={\%*}{*)},          % if you want to add LaTeX within your code
   extendedchars=true,              % lets you use non-ASCII characters; for 8-bits encodings only, does not work with UTF-8
   frame=single,	                   % adds a frame around the code
   keepspaces=true,                 % keeps spaces in text, useful for keeping indentation of code (possibly needs columns=flexible)
   keywordstyle=\color{blue},       % keyword style
   language=C++,                 % the language of the code
   otherkeywords={*,...},           % if you want to add more keywords to the set
   numbers=left,                    % where to put the line-numbers; possible values are (none, left, right)
   numbersep=5pt,                   % how far the line-numbers are from the code
   numberstyle=\tiny\color{mygray}, % the style that is used for the line-numbers
   rulecolor=\color{black},         % if not set, the frame-color may be changed on line-breaks within not-black text (e.g. comments (green here))
   showspaces=false,                % show spaces everywhere adding particular underscores; it overrides 'showstringspaces'
   showstringspaces=false,          % underline spaces within strings only
   showtabs=false,                  % show tabs within strings adding particular underscores
   stepnumber=2,                    % the step between two line-numbers. If it's 1, each line will be numbered
   stringstyle=\color{mymauve},     % string literal style
   tabsize=2,	                   % sets default tabsize to 2 spaces
   title=\lstname                   % show the filename of files included with \lstinputlisting; also try caption instead of title
 }

 %\lstinputlisting[linerange={78-90}]{../src/proj1.cc}


\section{Results and discussion}

As a check, we calculate
$\delta _0$ from the analytic formula

\begin{equation}
	\delta _0(E) = \arctan \left [ \sqrt{\frac{E}{E+V_0}} \tan \left ( R \sqrt{2 \mu (E + V_0)}
 \right ) \right ] - R \sqrt{2 \mu E}
	\label{eq:deltaAnalytic}
\end{equation}
where
\begin{equation*}
	E = \frac{k^2}{2 \mu}
\end{equation*}


%\begin{table}
%\centering
%	\begin{tabular}{ c | c c c }
%	 & \multicolumn{3}{c}{Time [$\mu$s]}\\
%	$n$ & General & Specialized & LU Decomp\\
%\hline
%	$10^{1}$ & 2       & 2       & 10 \\
%	$10^{2}$ & 7       & 5       & 3500 \\
%	$10^{3}$ & 62      & 50      & 1700000\\
%	$10^{4}$ & 600     & 555     & \\
%	$10^{5}$ & 6100    & 5100    & \\
%	$10^{6}$ & 32500   & 28300   & \\
%	$10^{7}$ & 244000  & 205000  & \\
%\hline
%	\end{tabular}
%	\caption{Summary of execution times for the algorithms used.
%	We note that the execution time can vary from run to run due
%	to variations in CPU load. The execution time also varies
%	from platform to platform.}
%	\label{tab:speedresults}
%\end{table}


%\begin{figure*}
%\centering
%	\includegraphics{figures/sols.pdf}
%	\caption{A comparison of the results from the numerical solution
%	(with various choices for the number of grid points) to the exact
%	result. The blue curve plots the numerical result with 10 grid
%	points, the green curve uses 100 grid points and the red curve
%	uses 1000 grid points. The black curve plots the exact result.
%	Good agreement with the exact result is obtained once 100 grid
%	points are used.}
%	\label{fig:compexact}
%\end{figure*}



\section{Conclusions}

%\begin{thebibliography}{99}
%\bibitem{miller2006} G.~A.~Miller, A.~K.~Opper, and E.~J.~Stephenson, Annu.~Rev.~Nucl.~Sci.~{\bf 56}, 253 (2006).
%\bibitem{Morten} M. Hjorth-Jensen, Computational Physics Lecture Notes Fall 2015, August 2015.
%\bibitem{Mslides} M. Hjorth-Jensen, Computational Physics Notes, compphysics.github.io
%\bibitem{Golub1996} G. Golub, C. Van Loan, \textit{Matrix Computations} (John Hopkins University Press, 1996)
%\end{thebibliography}

\bibliographystyle{plainnat}
\bibliography{refs}

\end{document}
