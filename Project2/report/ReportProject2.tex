\documentclass[10pt,showpacs,preprintnumbers,footinbib,amsmath,amssymb,aps,prl,twocolumn,groupedaddress,superscriptaddress,showkeys]{revtex4-1}
\usepackage{graphicx}
\usepackage{float}
\usepackage{dcolumn}
\usepackage{bm}
\usepackage[colorlinks=true,urlcolor=blue,citecolor=blue]{hyperref}
\usepackage{color}
\usepackage{listings}

\newcommand{\deriv}[3][]{% \deriv[<order>]{<func>}{<var>}
	\ensuremath{ \frac{d^{#1} {#2}}{d {#3}^{#1}} } }


\begin{document}
\title{Building and effective field theory for neutron-proton scattering}
\author{Thomas Redpath and Dayah Chrisman}
\affiliation{Department of Physics, Michigan State University}
\begin{abstract}

We build an effective field theory (EFT) to describe neutron-proton scattering. We take the
underlying theory to be a sum of three Yukawa potentials with empirically derived parameters
(the potential model from the previous project). By fitting to the low energy phase shifts
($\delta _0$) calculated from solving the Lippman-Schwinger equation, we determine the
coefficients for the contact interactions in the EFT potential. ...


\end{abstract}
\maketitle

\section{Introduction}

Measuring the final states of two nucleons after they interact provides information about
the nature of their interaction. One way to summarize the information we get from these
scattering experiments is through the phase shift - which can be roughly described as a
shift in the scattered wavefunction relative to the incoming one. In this project, we study
the behavior of the $l=0$ phase shift ($\delta _0$) as a function of energy for a toy
neutron-proton ($np$) interaction model.

Qualitatively, we say that a repulsive potential will result in a negative
phase shift - the scattered wavefunction is ``pushed out'' and that an attractive potential
will give a positive phase shift. Thus by analyzing experimental data we can deduce whether
a potential is repulsive, attractive, or a mixture of both. The nuclear potential is negligible at
long ranges ($r >$ a few fm), attractive at intermediate ranges ($r \sim 1$ fm) and repulsive
at very short distances ($r < 0.5$ fm). By examining the phase shift as a function of energy,
we can see at which energies the incoming particle is ``repelled'' (giving a negative phase
shift). This can give us an idea of the range of the repulsive inner part of the potential, since
the particle needs enough energy to approach the inner, repulsive part of the potential.

We organize this report as follows: first we summarize the algorithms we need to solve the
problem numerically. We then describe the framework we used to carry out our calculations.
Next, we present the results of our calcuations and some qualitative insights derived from
considering the variable phase approach. Finally, we summarize our findings and offer some
comments about possible extensions of our framework.



\section{Theory and algorithms}\label{sec:theory}

\subsection{Lippman-Schwinger Equation}

In order to calculate the phase shifts $\delta_l$, we need to solve the Schr{\"o}dinger equation
for the neutron-proton system with $E > 0$. This can be expressed as an integral equation for
the reaction matrix

\begin{equation}
\begin{split}
    R_l(k,k') &= V_l(k,k') + \\
  &\frac{2}{\pi}\hat{P}
                \int_0^{\infty}dqq^2V_l(k,q)\frac{1}{E-q^2/m}R_l(q,k'),
   \label{eq:ls1}
\end{split}
\end{equation}
where $E = k _0 ^2 / m$ is the total kinetic energy of the nucleons in the center-of-mass
system and the momentum-space representation of the potential $V(k',k)$ is used. We solve this
equation numerically on a discretized domain where we've set the mesh points and
weights using Gaussian Legendre quadrature. We map the points and weights from the interval of
the Legendre polynomials ($x \in [-1,1]$) to our desired interval $r \in [0,\infty)$ using the
following mapping

\begin{align*}
          k_i&=const\times tan\left\{\frac{\pi}{4}(1+x_i)\right\} \\
          \omega_i&= const\frac{\pi}{4}\frac{w_i}{cos^2\left(\frac{\pi}{4}(1+x_i)\right)}.
\end{align*}

We summarize the prescription for solving eq.~\ref{eq:ls1} numerically from
\citet{lectureNotes}. First, we employ Cauchy's principle-value prescription
to re-write eq.~\ref{eq:ls1} in the form

\begin{equation}
\begin{split}
    R(k,k') &= V(k,k') + \\
   & \frac{2}{\pi}
                \int_0^{\infty}dq
                \frac{q^2V(k,q)R(q,k')-k_0^2V(k,k_0)R(k_0,k')  }
                     {(k_0^2-q^2)/m},
   \label{eq:ls2}
\end{split}
\end{equation}
which is represented on the discretized mesh as

\begin{equation}
\begin{split}
R(k,k') &= V(k,k') +\frac{2}{\pi}\sum_{j=1}^N\frac{\omega_jk_j^2V(k,k_j)R(k_j,k')}{(k_0^2-k_j^2)/m}- \\
  & \frac{2}{\pi}k_0^2V(k,k_0)R(k_0,k')\sum_{n=1}^N\frac{\omega_n}{(k_0^2-k_n^2)/m},
\end{split}
\label{eq:ls3}
\end{equation}
where there are $N$ mesh points from the Gaussian quadrature mapping
and one point that is the chosen ``observable'' point corresponding to the
relative momentum for which we want to calculate the phase shift. In total
there are $(N+1)$ mesh points.

Next, we organize eq.~\ref{eq:ls3} as a matrix equation by defining a
matrix $A$

\begin{equation*}
	A_{i,j} = \delta _{i,j} - V(k_i,k_j)u_j
\end{equation*}
where

\begin{equation}
     u_j=\frac{2}{\pi}
         \frac{\omega_jk_j^2}{(k_0^2-k_j^2)/m}\hspace{1cm}
         j=1,N
\label{eq:uj}
\end{equation}
and

\begin{equation}
     u_{N+1}=-\frac{2}{\pi}
          \sum_{j=1}^N\frac{k_0^2\omega_j}{(k_0^2-k_j^2)/m}.
\label{eq:u0}
\end{equation}

We can now solve for the $R$ matrix by inverting the matrix $A$ and multiplying $V$
by $A^{-1}$. Finally, the $R(N+1,N+1)$ element is related to the phase
shift for $k_0$ by

\begin{equation}
	R(k_0,k_0)=-\frac{tan\delta}{mk_0}
	\label{eq:PhaseShift}
\end{equation}
Using this prescription and some model for the interaction (see next
subsection), we can calculate $\delta$ for scattering at some incident energy.
We describe the \texttt{C++} code used to make these calculations in
the Methods section.


\subsection{Potential Model}

We currently have coded two potentials that can be used in the phase shift
calculation. The first is a finite square well (eq.~\ref{eq:FSpot}) used to
benchmark the program against the analytical result. The second is a
parameterized $^1S_0$ neutron-proton interaction (eq.~\ref{eq:NPpot}).

\begin{equation}
	V = -V_0 \Theta(a - r)
	\label{eq:FSpot}
\end{equation}

\begin{equation}
	V(r)=V_a \frac{e^{-ax}}{x}+V_b \frac{e^{-bx}}{x}+V_c \frac{e^{-cx}}{x},
	\label{eq:NPpot}
\end{equation}
with $x=\mu r$, $\mu=0.7$ fm$^{-1}$ (the inverse of the pion mass),
$V_a=-10.463$ MeV and $a=1$, $V_b=-1650.6$ MeV and $b=4$ and
$V_c=6484.3$ MeV and $c=7$.

For both potentials, we find the momentum-space representations using
the Fourier-Bessel transform, we specialize to the case $l=0$ since we're
only aiming to calculate $\delta_0$. For the finite well potential we
have

\begin{align*}
	V_l(k,k')&= \int j_l(kr)V(r)j_l(k'r)r^2dr\\
	&= V_0 \int_0^a \frac{\sin(kr) \sin(k'r) r^2}{k k' r^2} dr\\
	&= \frac{V_0}{k k'} \int_0^a \sin(kr) \sin(k'r) dr\\
	&= \frac{V_0}{k k'} \frac{k' \sin(ka) \cos(k'a) - k \cos(ka) \sin(k'a)}{k^2 - k'^2}
\end{align*}
and
\begin{equation*}
	V(k,k) = \frac{V_0}{k^2} \left ( \frac{a}{2} - \frac{2 \sin (ka)}{4 k} \right )
\end{equation*}
in the case $k=k'$. For the $np$ model, the momentum space representation
is

\begin{equation*}
V (k,k') =\frac{V_{\eta}}{4\mu k k'}\ln\left[\frac{(\mu \eta)^{2}+(k+k')^{2}}
{(\mu\eta)^{2}+(k-k')^{2}}\right]
\end{equation*}


\subsection{$A$ Matrix}

Care must be taken when setting up the matrix $A$ to solve for the reaction
matrix. Pedantically, there are four sections of this matix that must be treated
separately.

\begin{enumerate}
	\item $A(i,j) = \delta _{i,j} - V(i,j) u_j$ for $i,j \in 1..N$
	\item $A(N+1,j) = -V(N+1,j) u_j$ for $j \in 1..N$, the last row
	\item $A(i,N+1) = -V(i,N+1) u_{N+1}$ for $i \in 1..N$, the last column
	\item $A(N+1,N+1) = 1 - V(N+1,N+1) u_{N+1}$ bottom right corner
\end{enumerate}


%\subsection*{The General Algorithm}

%\[
%\begin{bmatrix}
%	d_1 & a_1 & 0 & \dots & \dots & 0 \\
%	c_1 & d_2 & a_2 & 0 & \dots & 0 \\
%	0 & c_2 & d_3 & a_3 & 0 & \dots \\
%	\vdots & & \ddots & \ddots & \ddots & \vdots \\
%	\vdots & & & c_{n-2} & \ddots & a_{n-1}\\
%	0 & & & & c_{n-1} & d_n \\
%\end{bmatrix}
%~
%\begin{bmatrix}
%	u_1\\
%	u_2\\
%	u_3\\
%	u_4\\
%	\vdots\\
%	u_n\\
%\end{bmatrix}
%~
%=
%~
%\begin{bmatrix}
%	f_1\\
%	f_2\\
%	f_3\\
%	f_4\\
%	\vdots\\
%	f_n\\
%\end{bmatrix}
%\]

\section{Methods}

We implemented our numerical solution to the Lippman-Schwinger equation as a \texttt{C++} class.
This class consists of three two-dimensional data structures to hold the $V$, $A$ and $R$ matrices,
two one-dimensional data structures to hold the weights and momentum mesh points that define the
integration domain and several ancillary variables that specify the nucleon mass and other book-keeping
parameters. Application of the algorithms sketched in the previous section is carried out in a series of
member functions that (a) set up the mesh points and weights (b) set up the potential matrix (c) set
up the $A$ matrix (d) invert $A$ to get $R$ then extract $\delta _0$. In principle,
different functions may be used to set up different potentials; we coded one for the model $np$ interaction
given above and one for a finite spherical well to check that we recover the analytic result.
This class is defined in the source
files \texttt{NucleonScattering.cpp} and \texttt{NucleonScattering.hh}. An instance of the class is
created and the correct sequence of methods is called in the \texttt{main} function defined in
\texttt{main.cpp}. We include three versions of \texttt{main.cpp} that (1) calculate $\delta_0$
for a single energy, (2) loop over energies and calculate $\delta_0$ for the a range of energies
using the finite spherical potential and (3) run the calculation for the experimental energies.
The results from cases (2) and (3) are saved to the files \texttt{output/FiniteSpherel0ps.txt}
and \texttt{output/l0ps.txt} repectively. We generated FIG.~\ref{fig:analyticDelta} and
FIG.~\ref{fig:NPexp} from these files.


%% -------------------------- CODE LISTING -------------------------- %%

%% 
 \definecolor{mygreen}{rgb}{0,0.6,0}
 \definecolor{mygray}{rgb}{0.5,0.5,0.5}
 \definecolor{mymauve}{rgb}{0.58,0,0.82}

 \lstset{ %
   backgroundcolor=\color{white},   % choose the background color; you must add \usepackage{color} or \usepackage{xcolor}
   basicstyle=\footnotesize,        % the size of the fonts that are used for the code
   breakatwhitespace=false,         % sets if automatic breaks should only happen at whitespace
   breaklines=true,                 % sets automatic line breaking
   captionpos=b,                    % sets the caption-position to bottom
   commentstyle=\color{mygreen},    % comment style
   deletekeywords={...},            % if you want to delete keywords from the given language
   escapeinside={\%*}{*)},          % if you want to add LaTeX within your code
   extendedchars=true,              % lets you use non-ASCII characters; for 8-bits encodings only, does not work with UTF-8
   frame=single,	                   % adds a frame around the code
   keepspaces=true,                 % keeps spaces in text, useful for keeping indentation of code (possibly needs columns=flexible)
   keywordstyle=\color{blue},       % keyword style
   language=C++,                 % the language of the code
   otherkeywords={*,...},           % if you want to add more keywords to the set
   numbers=left,                    % where to put the line-numbers; possible values are (none, left, right)
   numbersep=5pt,                   % how far the line-numbers are from the code
   numberstyle=\tiny\color{mygray}, % the style that is used for the line-numbers
   rulecolor=\color{black},         % if not set, the frame-color may be changed on line-breaks within not-black text (e.g. comments (green here))
   showspaces=false,                % show spaces everywhere adding particular underscores; it overrides 'showstringspaces'
   showstringspaces=false,          % underline spaces within strings only
   showtabs=false,                  % show tabs within strings adding particular underscores
   stepnumber=2,                    % the step between two line-numbers. If it's 1, each line will be numbered
   stringstyle=\color{mymauve},     % string literal style
   tabsize=2,	                   % sets default tabsize to 2 spaces
   title=\lstname                   % show the filename of files included with \lstinputlisting; also try caption instead of title
 }

 %\lstinputlisting[linerange={78-90}]{../src/proj1.cc}


%\begin{figure*}
%	\includegraphics[width=\textwidth]{figures/stability.pdf}
%	\caption{Results for the finite square well ($V_0 = -0.5$ fm$^2$, $a = 1.$ fm) $l=0$
%	phase shifts  as a function of lab energy. In each plot, our calculations are represented by the
%	blue points and the analytic results are plotted in red. The black points in the lower plots
%	show the deviation of our calculation from the analytic result. From left to right we plot the
%	results of our calculation for N=5, N=50 and N=100 to show convergence for 50 mesh points.}
%	\label{fig:analyticDelta}
%\end{figure*}


\section{Results and discussion}

As a check, we calculate $\delta _0$ from the analytic formula for the simple finite square
well.

\begin{equation}
	\delta _0(E) = \arctan \left [ \sqrt{\frac{E}{E+V_0}} \tan \left ( R \sqrt{2 \mu (E + V_0)}
 \right ) \right ] - R \sqrt{2 \mu E}
	\label{eq:deltaAnalytic}
\end{equation}
where
\begin{equation*}
	E = \frac{k^2}{2 \mu}
\end{equation*}
The results are shown for three different numbers of mesh points ($N=5,N=50,N=100$) in
FIG.~\ref{fig:analyticDelta}. In terms of convergence, our result is stable with $N=50$ points.
The overall agreement with the analytic result is worse for higher lab energies because there
are fewer mesh points at these higher energies. The way we map our integration points from
the interval $[-1,1]$ to $[0,\infty)$ results in a higher density of mesh points at lower energies.
Therefore, we expect the calculation to be more accurate at lower energies. FIG.~
\ref{fig:analyticDelta}
shows that our results agree with the analytic value to within 25\% for $E_{\mathrm{lab}}$ less
than 100 MeV.

In FIG.~\ref{fig:NPexp}, we plot our calculated phase shifts next to the experimental results
tabulated in \citep{Nijmegen}. For lab energies at and below 100 MeV, we see good agreement
with the data ($< 10$\%), suggesting that our model potential best reproduces the underlying
physics at lower energies. The model also reproduces the point ($E_{\mathrm{lab}} = 250$ MeV)
where $\delta_0$ turns negative and we start to probe the repulsive core of the interaction.


\subsection{Variable Phase Approach}
The variable phase approach (VPA) relies on numerical methods to solve the differential equation.

\[
\frac{d\delta(k,r)}{dr} = -\frac{1}{k} 2M V(r) \sin^2[kr + \delta(k,r)]
\]
for $\delta(k,r)$. The VPA is simple to use,
and information about the potential is relatively easy to extract. For example, here we show
that the presence of bound states (and the number of them) for a certain potential depth can
be quickly determined, and short-range repulsive components of the potential can
be inferred based on phase shift calculations at various energies.

The VPA truncates a realistic potential at some cutoff radius $\rho$.
Then, we inegrate over $r$ out to a sufficiently large radius ($R_{\mathrm{max}} > \rho$). 


In FIG.~\ref{fig:VPA}, we note that the phase shift becomes constant after a radius of 1;
this is the radius of the square well potential we used. Because of the discontinuity in the potential at the
edge of the square well, there is no ambiguity about how large we need to make our integration
radius: obviously, as soon as we are looking beyond the radius of potential, the phase shift will
converge to a constant value. This tells us that an integration radius greater than the square well 
will be sufficient for the phase shift to converge.

We then examined the plots of $k\cot(\delta)$ and $\delta$ to test Levinson's Theorem,
which states that the number of bound states can be calculated by looking at $\delta(E=0)$
and the limit of $\delta$ as $E$ approaches $\infty$

\begin{equation*}
	\phi_l(0)-\phi_l(\infty)=n\pi
\end{equation*}
where $n$ is the number of bound states the potential can hold. For a well depth $V_c=1.0$,
$\phi_l(0)=0$,  $\phi_l(\infty)=0$. This means that this well cannot bind any states
with $l=0$. For a well depth $V_c=2.0$, $\phi_l(0)=3.14$,  $\phi_l(\infty)=0$,
meaning that at this depth, the well can hold one $l=0$ bound state.

We add now a short-range repulsive potential and calculate the phase shifts using the VPA.The
sign of the phase shift can characterize a potential as attractive ($\delta_0>0$) or repulsive
($\delta_0<0$). Adding a repulsive term should begin to make a difference at higher
energies where the approaching particles can get close enough to ``see'' the effect of
the repulsive core. At these higher energies, we should see the phase shift turn negative,
indicating the repulsive character of the potential.

With $V_c=0$ there is no repulsive part and the potential is purely attractive, we get that
$\phi_l(\infty)=0$. This indicates, as we predicted, a purely attractive potential.
Around $V_c=18$, we can see the phase shift turning over at around $k=10$. As we increase
the height of the repulsive potential, the phase shifts start to become negative at lower and
lower energies, until at $V_c=100$ the phase shift turns negative at only $k=6$. 

Because the nuclear force is repulsive at short distances ($V \rightarrow \infty$ for $r \rightarrow 0$),
we can use calculated phase shifts from experimental cross sections to estimate the range
of the repulsive part of the nuclear force ($R_c$). The incoming momenta/energy ($k$) at which the
phase shift changes gives us an estimate of the length scale at which the repulsive piece of the
interaction becomes dominant.



\section{Conclusions}

In this project, we solved the Lippmann-Schwinger equation numerically to compute
the $l=0$ phase shifts for a model $np$ potential. Our calculations are stable with
$N=50$ mesh points. We see decent agreement with experimental data suggesting
that our model does a reasonable job of describing the underlying physics. Furthermore,
we explored the variable phase approximation to gain an intuitive understanding of
the relationship between the sign of $\delta_0$ and the sign of the potential. The
VPA also provides us with a simple framework to demonstrate Levinson's theorem.

We can easily incorporate any potential model into this framework by coding
the appropriate form of the potential. In this way, we could extend our analysis
to higher values of $l$ or other models for the interaction.

%\begin{thebibliography}{99}
%\bibitem{miller2006} G.~A.~Miller, A.~K.~Opper, and E.~J.~Stephenson, Annu.~Rev.~Nucl.~Sci.~{\bf 56}, 253 (2006).
%\bibitem{Morten} M. Hjorth-Jensen, Computational Physics Lecture Notes Fall 2015, August 2015.
%\bibitem{Mslides} M. Hjorth-Jensen, Computational Physics Notes, compphysics.github.io
%\bibitem{Golub1996} G. Golub, C. Van Loan, \textit{Matrix Computations} (John Hopkins University Press, 1996)
%\end{thebibliography}

\bibliographystyle{plainnat}
\bibliography{refs}

\end{document}
